\documentclass[handout]{ximera}
\input{preamble}

\title{The Chain Rule}

\begin{document}

\begin{abstract} Introduction %Video 4 Introduction
\end{abstract}

\maketitle

On the next pages, you will watch two videos about the chain rule and will then answer some questions about the video.

\begin{itemize}
\item The goals of these videos are to explain when you would need to use the chain rule and how to use the chain rule to find derivatives
\item You use the chain rule when you have two composed functions - one function ``inside'' another, like $f(g(x))$
\item To find the derivative, you do $f'(g(x)) * g'(x)$
\item The reason for doing this is because the derivative of $f$ doesn’t just depend on $x$, but rather on the value of $g(x)$. So when $g(x)$ changes quickly, it affects how quickly $f(g(x))$ changes
\end{itemize}

\end{document}